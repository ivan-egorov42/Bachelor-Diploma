\section{Постановка задачи}
\label{sec:Chapter1} \index{Chapter1}

% Необходимо формально изложить суть задачи в данной секции, предоставив такие
% ясные и точные описания, которые позволят в последующем оценить, насколько
% разработанное решение соответствует поставленной задаче. Текст главы должен
% следовать структуре технического задания, включая как описание самой задачи,
% так и набор требований к ее решению.

Целью данной работы является разработка и реализация алгоритма векторизации линейного кода, содержащего ленивые вычисления.

Для достижения данной цели были поставлены следующие задачи:

\begin{itemize}
    \item Провести анализ частоты встречаемости ленивых вычислений, которые можно было бы векторизовать
    \item Исследовать подходы к векторизации в современных компиляторах
    \item Изучить работу компилятора с памятью при векторизации данных и инструкций
    \item Реализовать поиск подходящих шаблонов, содержащих ленивые вычисления, и отбор потенциальных кандидатов для векторизации
    \item Реализовать версионирование кода в местах применения трансформации
    \item Интегрировать оптимизацию в качестве отдельного прохода в компиляторе GCC
    \item Провести замеры и добиться повышения производительности на реальных тестовых данных
\end{itemize}

Результат работы можно считать успешным, если удастся показать повышение производительности на реальных тестовых данных. 

Актульность работы заключается в том, что прежде всего разработка оптимизаций является основополагащей задачей в современных оптимизирующих компиляторах и создание новых алгоритмов векторизации, повышающих производительность, представляет научный и производственный интерес. Важным аспектом также является то, что метод векторизации ленивых вычислений, представляемый в работе, позволяет решить проблему векторизации условных выражений, расширив функционал современных компиляторов.
\newpage
