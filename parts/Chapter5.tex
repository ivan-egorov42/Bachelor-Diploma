\section{Заключение}
\label{sec:Chapter5} \index{Chapter5}

% Здесь надо перечислить все результаты, полученные в ходе работы. Из текста
% должно быть понятно, в какой мере решена поставленная задача.

\subsection{Результаты}

В качестве результатов и демонстрации ценности проделанной работы приведена статистика работы реальных тестовых данных из пакетов СPUBench и SPEC CPU 2017. 

Данная оптимизация повышает производительность путем уменьшения количества косвенных переходов в коде. Максимальное же ускорение достигается при наличии таких конструкций внутри горячих участков кода. В качестве тестовых данных были выбраны задачи пакетов SPEC CPU 2017 и CPUBench. Так, например наибольшее ускорение в 3.5\% было достигнуто на бенчмарке gzip из пакета CPUBench, содержащем в себе вложенный цикл с подобным шаблоном. Ошибка предсказаний переходов в самой функции с циклом упала на 60\%. Всего было найдено и оптимизировано 88 подобных конструкций среди бенчмарков пакета CPUBench. Также наблюдалось ускорение на тестах tpcc и gcc. На тестах пакета SPEC CPU 2017 ускорения не наблюдалось, однако деградации производительности также отсутствовали. Результы тестирования представлены в таблице ниже.

\begin{table}[!htb]
    \centering
    \begin{tabular}{|c|c|c|c|}
        \hline
           & Без оптимизации (сек.)  & С оптимизацией (сек.)  & Улучшение  \\ \hline
        gzip    & 178.1  & 172.5  & 3.5\%  \\ \hline
        tpcc    & 127.9 & 125.5 & 1.9\% \\ \hline
        gcc     & 226.4 & 224.5 & 0.8\% \\ \hline
        python  & 175.4 & 174.9 & 0.3\% \\ \hline
        tpch    & 117.2 & 116.8 & 0.3\% \\ \hline
        xz      & 218.7 & 218.2 & <0.3\% \\ \hline
        velvet  & 91.2 & 91.0 & <0.3\% \\ \hline
    \end{tabular}
\end{table}

Замеры проводились на сервере с архитектурой \textit{aarch64}.

Базовая конфигурация опций компилятора: -О3, -flto.

\subsection{Выводы}

В рамках данной работы была рассмотрена оптимизация векторизации ленивых вычислений. Были рассмотрены существующие решения и был предложен алгоритм векторизации с учетом особенностей страничной организации памяти. Выбранный алгоритм был реализован и внедрен в компилятор GCC.

Было показано, что оптимизация дает ускорение работы целевых тестовых данных. Так же было показано, что ускорение достигается за счет уменьшения количества косвенных переходов в коде. Результат работы можно считать успешным, так как достигнуты все цели и задачи.

\newpage
